\documentclass[a4paper,12pt]{scrartcl}
\usepackage{hyperref}
\usepackage{bibentry}
\usepackage{amssymb}
\usepackage{graphicx}
\newcommand{\figwidth}{0.49\linewidth}
\title{Preferential Rewiring Model in Opinion Formation}
\author{Mathis Antony}

\begin{document}
\maketitle
\section{Model}
As discussed last week, an Agent chosen to update either rewires a link (with probability $\phi$) or updates his opinion (with probability $1-\phi$). The majority is determined in one of the following two ways:
\begin{itemize}
\item \textbf{unweighted}: every neighbors opinion carries equal weight
\item \textbf{weighted}: every neighbors opinion has a weight proportional to the degree of said neighbor
\end{itemize}
We have homophilic respectively heterophilic agents who rewire to agents with the same respectively different opinions. The density of heterophilic agents is $\eta$. The rewiring is done preferentially like in the generative process for BA networks: the probability that an node (or agent) receives the new link is proportional to the degree of the node. Initially we use a BA network with 100 Agents and attachment parameter $m=3$\footnote{Here this is actually an important parameter. If it is even, there will be a more tie situation than when it is odd. Opinions therefore change less often and the consensus time is expected to increase. Some preliminary results suggest that the consensus time is a bout an order of magnitude higher when $m=2$ as opposed to $m=3$.}.

\section{Results}
\subsection{Average Consensus Time}
\begin{figure}
\centering
\includegraphics[width=\figwidth]{../code/rewi/fig/grid_weighted.pdf}
\includegraphics[width=\figwidth]{../code/rewi/fig/grid_unweighted.pdf}
\includegraphics[width=\figwidth]{../code/rewi/fig/grid_diff.pdf}
\caption{Average consensus time for the two majority rules and the relative difference between the weighted and unweighted rule.}
\label{fig:1}
\end{figure}
Figure \ref{fig:1} shows that the average consensus time is significantly lower with the weighted majority update rule. For small values of $\phi$ the difference is the most prominent. There the consensus time is up to $40\%$ less due to the weighted update rule.

In general we also observe that the density of heterophilic agents marginally decreases the consensus time for both majority rules.

%
%
%
%\subsection{Consensus Time Distribution}
%\begin{figure}
%\centering
%\includegraphics[width=\figwidth]{../code/rewi/fig/ecdf_phi0.pdf}
%\includegraphics[width=\figwidth]{../code/rewi/fig/ecdf_phi0_15.pdf}
%\includegraphics[width=\figwidth]{../code/rewi/fig/ecdf_phi0_475.pdf}
%\caption{Empirical cumulative distribution function of the consensus time. Each curve is made from 1000 trials.}
%\end{figure}
%
%\subsection{Consensus Time Degree Distribution}
%\begin{figure}
%\centering
%\includegraphics[width=\figwidth]{../code/rewi/fig/Deg_Dist_phi0_1.pdf}
%\includegraphics[width=\figwidth]{../code/rewi/fig/Deg_Dist_phi0_5.pdf}
%\includegraphics[width=\figwidth]{../code/rewi/fig/Deg_Dist_phi0_9.pdf}
%\caption{Consensus Time degree distribution.}
%\end{figure}
%
%\subsection{Consensus Time Degree Distribution by Type}
%\begin{figure}
%\centering
%\includegraphics[width=\linewidth]{../code/rewi/fig/Deg_Dist_Type_phi0_1.pdf}
%\includegraphics[width=\linewidth]{../code/rewi/fig/Deg_Dist_Type_phi0_5.pdf}
%\caption{Consensus time degree distribution for the two types of agents.}
%\end{figure}


\bibliographystyle{unsrt}
\bibliography{../papers/refs/refs.bib}
\end{document}
